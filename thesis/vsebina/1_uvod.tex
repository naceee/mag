\section{Uvod}
\label{sec:uvod}

Optimizacija je področje matematike in računalništva, ki se ukvarja z iskanjem najboljše rešitve glede na določen kriterij. Pojavlja se na veliko področjih, na primer v inženirstvu, ekonomiji in drugih. Pri reševanju praktičnih problemov pa pogosto ne želimo optimirati le enega kriterija, ampak več, ponavadi nasprotujočih si kriterijev. Kadar, na primer, iščemo hotel ob morju, si želimo hkrati minimirati tako ceno, kot tudi oddaljenost od morja, kar pa pogosto ni moč doseči. Namesto ene optimalne rešitve tako iščemo množico rešitev, ki predstavljajo različne kompromise med kriteriji. Takšnim problemom pravimo večkriterijski optimizacijski problemi in jih rešujemo z uporabo večkriterijskih optimizacijskih algoritmov~\cite{Deb2015}.

Obstaja veliko večkriterijskih optimizacijskih algoritmov, ki se razlikujejo predvsem po strategiji preiskovanja prostora rešitev in načinu primerjave med rešitvami. V večdimenzionalnem prostoru kriterijev namreč to ni enoznačno. Veliko algoritmov za rangiranje rešitev uporablja nedominirano urejanje (angl.~\textit{nondominated
sorting}), ki rešitve razdeli na zaporedne fronte, glede na relacijo dominiranosti med rešitvami~\cite{nsga2}. Pri algoritmu NSGA-II se rešitve znotraj posamezne fronte rangirajo glede na zgoščenost (angl.~\textit{crowding distance})~\cite{nsga2}, pri algoritmu NSGA-III pa glede na oddaljenost do referenčnih vektorjev~\cite{nsga3}. Drugače je pri algoritmu SMS-EMOA, ki za rangiranje rešitev uporablja prispevek hipervolumna (angl.~\textit{hypervolume contribution}) posamezne rešitve k vrednosti hipervolumna celotne množice rešitev~\cite{smsemoa}.  

Večkriterijska evolucijska strategija s prilagajanjem kovariančne matrike in selekcijo z vejico (angl.~\textit{Comma-Selection Multiobjective Covariance Matrix Adaptation Evolution Strategy} - COMO-CMA-ES)~\cite{toure-como-cma-es} je algoritem za reševanje dvokriterijskih optimizacijskih problemov, ki prostor preiskuje z algoritmom CMA-ES~\cite{cmaes}, za primerjavo med rešitvami pa uporablja izpopolnjeno različico hipervolumna~\cite{definitions}, UHV (angl.~\textit{Uncrowded Hypervolume}). V študiji~\cite{toure-como-cma-es} COMO-CMA-ES dosega obetavne rezultate na testnih optimizacijskih problemih. 

V izpopolnjeni različici hipervolumna UHV izračuna algoritem za dano rešitev razdaljo do nedominiranega območja. Avtorji COMO-CMA-ES so ta algoritem zasnovali le za dva kriterija, torej v dveh dimenzijah. 

V magistrskem delu predstavimo nov algoritem za računanje razdalje do nedominiranega območja, imenujemo ga ARRNO. Deluje za poljubno dimenzijo $D \geq 3$, kar omogoča razširitev algoritma COMO-CMA-ES na poljubno število kriterijev. Nov algoritem ARRNO implementiramo in testiramo, da pokažemo njegovo pravilnost ter hitrost delovanja v različnih scenarijih in dimenzijah. 

Implementacija algoritma ARRNO za tri in štiri dimenzije je že na voljo tudi kot del knjižnice  \texttt{moarchiving}~\cite{moarchiving}, v programskem jeziku Python. Knjižnica omogoča hranjenje arhiva nedominiranih rešitev večkriterijskih optimizacijskih problemov, hitro računanje hipervolumna ter drugih indikatorjev in pomožnih funkcij, med katerimi je tudi računanje razdalje do nedominiranega območja. 

\subsection{Pregled področja}
V literaturi povezani z večkriterijsko optimizacijo je opisanih veliko sorodnih geometrijskih problemov in algoritmov za njihovo reševanje, ki se večinoma osredotočajo na problem učinkovitega računanja hipervolumna. V~\cite{vector_maxima} je predstavljen algoritem deli in vladaj za iskanje nedominiranih točk v poljubno dimenzionalnem prostoru, v~\cite{hv_complexity} pa se avtorji ukvarjajo z zahtevnostjo računanja hipervolumna, ter zanjo predstavijo nekatere spodnje ter zgornje meje. Bringmann v~\cite{Bringmann} pokaže, da je računanje hipervolumna v poljubni dimenziji lažje od računanja mere unije osi vzporednih enotskih kock, s čimer poda zgornjo mejo $O(n^{\left \lfloor d/2 \right \rfloor} \text{polylog}(n))$ za računanje hipervolumna.

V~\cite{Guerreiro} avtorji predstavijo hiter algoritem za računanje hipervolumna v treh in štirih dimenzijah (HV3D+, HV4D+), kjer uporabljajo kompleksne podatkovne strukture in rekurziven algoritem pometanja. Algoritem za računanje hipervolumna v treh dimenzijah implementirajo s časovno zahtevnostjo $O(n \log n)$, glede na število nedominiranih točk $n$, kar doseže teoretično spodnjo mejo iz~\cite{hv_complexity}. Algoritem za računanje hipervolumna v štirih dimenzijah pa ima časovno zahtevnost $O(n^2)$. Za oba algoritma avtorji pokažejo, da sta trenutno najhitrejša izmed znanih implementacij, s testiranjem na množicah točk različnih oblik in velikosti. 

Za razliko od računanja hipervolumna pa je razdalja do nedominiranega območja novejši in manj znan koncept. V dveh dimenzijah je algoritem za računanje razdalje do nedominiranega območja relativno preprost, avtorji algoritma COMO-CMA-ES ga implementirajo v~\cite{moarchiving}. Implementacije ali ideje za algoritem, ki bi deloval v več kot dveh dimenzijah, pa ne zasledimo. Nov algoritem ARRNO, ki ga vpeljemo v tem delu, se zgleduje tako po obstoječi implementaciji algoritma v dveh dimenzijah, kot tudi po algoritmih HV3D+ in HV4D+. Iz algoritma v dveh dimenzijah prevzame idejo računanja razdalje do vpetih točk, iz algoritmov HV3D+ in HV4D+ pa idejo o uporabi rekurzivnega algoritma pometanja.

\subsection{Struktura dela} 
V poglavju \ref{sec:definicija} predstavimo večkriterijsko optimizacijo in definiramo osnovne pojme. V poglavju \ref{sec:resevanje} najprej definiramo problem računanja razdalje med točko in nedominiranim območjem in predstavimo obstoječ algoritem za njegovo reševanje v dveh dimenzijah. Nato predstavimo nov algoritem za reševanje problema v treh dimenzijah ter ga razširimo na več dimenzij. V poglavju~\ref{sec:analiza_pravilnosti} predstavimo metodo, s katero testiramo pravilnost algoritma ter rezultate testiranja. V poglavju~\ref{sec:racunska_zahtevnost} pa najprej teoretično analiziramo časovno in prostorsko zahtevnost algoritma, nato pa časovno zahtevnost testiramo tudi na primerih različnih dimenzij in velikosti ter rezultate prikažemo v obliki grafov. V poglavju~\ref{sec:zakljucek} rezultate povzamemo in predstavimo ideje za prihodnje delo. 