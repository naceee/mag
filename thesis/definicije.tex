% naložite dodatne pakete, ki jih potrebujete
\usepackage{units}        % fizikalne enote kot \unit[12]{kg} s polovico nedeljivega presledka, glej primer v kodi
\usepackage{graphicx}     % za slike
\usepackage{tikz}
\usepackage{tikz-3dplot}
\usepackage{subcaption}
\usepackage{amssymb}
\usepackage{ marvosym }
% VEČ ZANIMIVIH PAKETOV
% \usepackage{array}      % več možnosti za tabele
% \usepackage[list=true,listformat=simple]{subcaption}  % več kot ena slika na figure, omogoči slika 1a, slika 1b
% \usepackage[all]{xy}    % diagrami
% \usepackage{doi}        % za clickable DOI entrye v bibliografiji
% \usepackage{enumerate}     % več možnosti za sezname

% Za barvanje source kode
% \usepackage{minted}
% \renewcommand\listingscaption{Program}

% Za pisanje psevdokode
\usepackage[noend]{algpseudocode}  % za psevdokodo
\usepackage{algorithm}
\floatname{algorithm}{Algoritem}
\renewcommand{\listalgorithmname}{Kazalo algoritmov}

\makeatletter
\renewcommand{\alglinenumber}[1]{\footnotesize\textcolor{gray}{#1}}
\makeatother

% deklarirajte vse matematične operatorje, da jih bo LaTeX pravilno stavil
% \DeclareMathOperator{\...}{...}

% vstavite svoje definicije ...
\newcommand{\R}{\mathbb R}
\newcommand{\N}{\mathbb N}
\newcommand{\Z}{\mathbb Z}
% Lahko se zgodi, da je ukaz \C definiral že paket hyperref,
% zato dobite napako: Command \C already defined.
% V tem primeru namesto ukaza \newcommand uporabite \renewcommand
\newcommand{\C}{\mathbb C}
\newcommand{\Q}{\mathcal Q}
\newcommand{\V}{\mathcal V}
\renewcommand{\P}{\mathcal P}
\renewcommand{\sp}{S_{\text{p}}}
\newcommand{\sv}{S_{\text{v}}}

% vsako poglavje na svoji strani
\AddToHook{cmd/section/before}{\clearpage}
% todoji z rdečo barvo
\usepackage[dvipsnames]{xcolor}
\newcommand{\todo}[1]{{\color{RubineRed}TODO: #1}}
\newcommand{\tea}[1]{\textcolor{Orange}{[Tea]: #1}}
\newcommand{\nace}[1]{\textcolor{Blue}{[Nace]: #1}}
\newcommand{\sergio}[1]{\textcolor{ForestGreen}{[Sergio]: #1}}

\definecolor{fillcol}{RGB}{254,227,145}
\definecolor{nodecol}{RGB}{204,76,2}
\definecolor{nodecol2}{RGB}{254,153,41}

\usepackage{enumitem}

\usepackage{pgfplots}
\usepackage{pgfplotstable}
\usepgfplotslibrary{colorbrewer}
\pgfplotsset{compat=1.18}

\DeclareFontFamily{U}{mathb}{\hyphenchar\font45}
\DeclareFontShape{U}{mathb}{m}{n}{
<-6> mathb5 <6-7> mathb6 <7-8> mathb7
<8-9> mathb8 <9-10> mathb9
<10-12> mathb10 <12-> mathb12
}{}
\DeclareSymbolFont{mathb}{U}{mathb}{m}{n}
\DeclareMathSymbol{\llcurly}{\mathrel}{mathb}{"CE}
\DeclareMathSymbol{\ggcurly}{\mathrel}{mathb}{"CF}
